\chapter{Продвижение времени} \label{chapt1}

\section{Имитационное моделирование} \label{sect1}

\todo{Шеннон о моделях. Окольнишников о имитационном моделировании}

Благодаря интенсивному развитию вычислительной техники моделирование приобретает общенаучный характер и применяется в исследованиях объектов и процессов, происходящих в природе, в науках о человеке и обществе.

Модель является представлением объекта, системы или понятия в некоторой форме, отличной от формы их реального существования. Модель служит средством, помогающим в объяснении, понимании или совершенствовании системы. Модель какого-либо объекта может быть или точной копией объекта или отображать некоторые характерные свойства объекта в абстрактной форме.

Моделирование применяется сегодня в самых различных областях: экологии и геофизике (анализ распространения загрязняющих веществ в атмосфере), транспорте (конструирование транспортных средств, полетные имитаторы для тренировки пилотов), электронике и электротехнике (эмуляция работы электронных устройств), экономике и финансах (прогнозирование цен на финансовых рынках), архитектуре и строительстве (исследование поведения зданий, конструкций и деталей под механической нагрузкой), управлении и бизнесе (моделирование рынков сбыта и рынков сырья), промышленности (моделирование роботов и автоматических манипуляторов), медицине и биологии (моделирование пандемий и эпидемий), политике и военном деле (моделирование развития межгосударственных отношений, моделирование театра военных действий).

Одним из наиболее важных применений моделей в практическом, и в историческом аспектах является прогнозирование поведения моделируемых объектов.

Применение моделей позволяет проводить контролируемые эксперименты в тех ситуациях, где экспериментирование на реальных объектах было бы практически невозможным или экономически нецелесообразным.~\cite{Shennon} В таких случаях может быть построена модель, на которой необходимые эксперименты могут быть проведены с относительной легкостью и недорого.

%Непосредственное экспериментирование с системой обычно состоит в варьировании её некоторых параметров. При этом, поддерживая все остальные параметры неизменными, наблюдают результаты эксперимента.  

%Зачастую моделирование используется в тех случаях, когда исследование процесса невозможно или невыгодно по определенным причинам, например, это может быть высокая стоимость создания процесса, его сложное строение, небольшие пространственно-временные размеры и т.д.



Выделяют три основных вида моделирования: аналитическое, численное и имитационное \cite{disksobmod}. Имитационное моделирование (ИМ) \nom[]{ИМ}{Имитационное моделирование} --- моделирование, при котором моделирующий алгоритм с той или иной степень точности воспроизводит функционирование исходной системы. Имитационная модель воспроизводит поведение моделируемой системы во времени. Имитационная модель может быть выполнена последовательно или параллельно.

Последовательное ИМ (еще его называют дискретным) характерно для выполнения модели на автономном однопроцессорном компьютере. Основным признаком последовательного ИМ является наличие централизованного списка событий и глобальных часов модельного времени. Организация выполнения имитационной модели в последовательном имитационном моделировании как правило состоит из следующих действий:
\begin{enumerate}
	\item Происходит активизация объектов для выполнения событий, запланированных на текущее значение часов модельного времени. Выполненные события удаляются из списка.
	\item В список событий включаются новые события, запланированные активными объектами. Такие события включаются в список событий вместе со значением модельного времени, в которое это событие должно быть выполнено в будущем.
	\item Происходит увеличение значения часов модельного времени, если на текущее значение часов модельного времени не осталось невыполненных событий. Затем следует переход на п.~1.
\end{enumerate}

Имеется несколько алгоритмов продвижения модельного времени в дискретном ИМ.
\begin{enumerate}
\item Event driven --- моделирование, управляемое событиями. В этой, наиболее распространенной реализации, в качестве следующего значения часов модельного времени выбирается минимальное время событий из списка событий. В целях оптимизации этого алгоритма список событий упорядочивается в порядке возрастания значений модельного времени, в которые события должны быть выполнены.
\item Time stepped --- моделирование с фиксированным шагом. Такое моделирование характеризуется тем, что значение модельного времени каждый раз увеличивается на фиксированную величину. Этот подход удобен при наличии условных событий, т.е. событий для выполнения которых требуется истинность некоторого логического условия. При управляемом событиями моделировании можно <<перескочить>> момент модельного времени, при котором условие истинно.
\item Wallclock time driven --- моделирование, управляемое часами реального времени. При таком моделировании значение часов модельного времени определяется некоторой неубывающей функцией от значений аппаратно или программно реализованных часов реального времени. Такие имитационные модели обычно связаны либо с аппаратурой, либо с людьми. Примером последних являются тренажеры, модельное время для которых определяется линейной функцией от реального времени.	
\end{enumerate}

\todo{http://www.scpe.org/index.php/scpe/article/view/221}

Под распределенным ИМ понимают распределенное выполнение единой программы имитационной модели на мультипроцессорной или мультикомпьютерной системе.

Последовательная имитационная модель может быть выполнена на параллельной вычислительной технике. При этом выигрыш по времени исполнения может быть достигнут за счет распараллеливания выполнения событий, запланированных на один и тот же момент модельного времени. Распределенное моделирование использует другую, более общую форму параллелизма, а именно параллельное выполнение событий, запланированных в различных отрезках модельного времени \cite{okol}.

При распределенном моделировании в качестве первичной единицы принимают логический процесс --- последовательную подмодель, структура которой состоит из управляющей программы и списка выполняющихся событий, запланированных на текущее значение часов модельного времени. Каждый логический процесс имеет собственный локальный список событий и собственные часы локального модельного времени. Логические процессы взаимодействуют исключительно с помощью передачи сообщений.
При этом общих для всей распределенной модели часов модельного времени и глобального списка событий явно не существует, так как наличие общей управляющей программы, работающей с этими глобальными структурами, было бы узким местом для параллельного исполнения.
Текущее модельное время всей модели в каждый момент времени равно $T_i=\min t(x_i^n)$. Логический процесс взаимодействует с другими процессами, передавая им сообщения. Например, логический процесс $x^m$ передает сообщение процессу $x^n$. Процесс $x^n$ получив сообщение, которое имеет форму события, вставляет это событие в упорядоченный список своих локальных событий в соответствии со значением модельного времени. Управляющая программа начинает выполнять модифицированный список локальных событий. Таким образом, полученное сообщение может изменить логику выполнения логического процесса-получателя. Передача сообщений может осуществляться логическими процессами непосредственно с помощью средств операционной системы. Но наиболее распространена схема, когда существует коммуникационная подсистема в качестве отдельного компонента. В этом случае логические процессы взаимодействуют с коммуникационной подсистемой с помощью определенного интерфейса. Это дает определенные преимущества:
\begin{enumerate}
\item Cистемная часть, реализующая передачу сообщений, сосредоточена в одной программе --- коммуникационной подсистеме.
\item Логические процессы могут быть реализованы на разных языках программирования.  В данном случае, если требуется перенос моделей и системы моделирования на другую вычислительную систему достаточно перенести коммуникационную подсистему.
\item Коммуникационная подсистема может синхронизировать выполнение логических процессов в модельном времени.
\end{enumerate}

Необходимость такой синхронизации можно рассмотреть на примере модели исторического события битвы при Ватерлоо~\cite{napoleon}, рисунок~\ref{fig:Корректная временная диаграмма исполнения модели}.

\begin{figure}[!ht]
\centering
\includegraphics[scale=1]{images/waterloo.pdf}
\caption{Модель <<Ватерлоо>>. Корректная временная диаграмма исполнения модели}
\label{fig:Корректная временная диаграмма исполнения модели}
\end{figure}

Наполеон Бонапарт (процесс $N$) должен был вступить в бой с герцогом Веллингтоном (процесс $W$), расположившемся со своей английской армией на пути к Брюсселю. Но оба полководца ждали подкрепления: герцог Веллингтон --- прусского князя Блюхера (процесс $B$), Наполеон --- своего маршала Груш\'и (процесс $G$). И Наполеон и Веллингтон отправили командующим своими подкреплениями сообщения с категорическим требованием скорейшего прибытия под Ватерлоо для усиления основной группировки войск. Тем временем, армия Бонапарта в ожидании прибытия подкрепления маршала Груши начала атаку неприятеля. Веллингтону ничего не оставалось, как принять бой. Не смотря на более выгодное расположение войск Веллингтона и примерно равные силы с Наполеоном, французы развивали успешное наступление на англичан и едва не опрокинули их --- положение Веллингтона становилось критическим. И только вовремя подоспевшее подкрепление князя Блюхера помогло герцогу сдержать натиск Наполеона, а затем и перейти в контрнаступление на французов. После чего, так и не дождавшийся своего маршала с подкреплением, французский император был разбит.

При распределенной реализации модельное время в разных логических процессах движется с разными скоростями.  Поэтому через некоторый интервал астрономического времени от начала исполнения модели модельные времена разных логических процессов могут оказаться разными.

Например, сообщение, посланное логическим процессом $B$, может быть получено логическим процессом $W$, как показано на рисунке~\ref{fig:Корректная временная диаграмма исполнения модели}, или как на рисунке~\ref{fig:Некорректная временная диаграмма исполнения модели}, если модельное время логического процесса $B$ <<отстает>>.

\begin{figure}[!ht]
\centering
\includegraphics[scale=1]{images/waterloo-kas.pdf}
\caption{Модель <<Ватерлоо>>. Некорректная временная диаграмма исполнения модели}
\label{fig:Некорректная временная диаграмма исполнения модели}
\end{figure}

<<Опоздание>> сообщения от процесса $B$  процессу $W$ вызвано не физическими, а <<техническими>> причинами. Например, процессор, на котором исполняется логический процесс $B$ перегружен или для определения ответа требуется слишком большой объем вычислений. Для моделирование такое отставание означает изменение прошлого, в момент времени $t4$, что означает нарушение правильности воспроизведения событий в моделируемой системе.

\subsection{Алгоритмы синхронизации модельного времени}

Для того, чтобы модель правильно воспроизводила последовательность событий в моделируемой системе, необходимо чтобы не возникало подобных парадоксов времени.
Тот процесс продвигает свое время вперед, который получил сообщение от процесса с б\'ольшим временем. Если процесс получает сообщение от процесса с меньшим временем, т.е. $t_i^m<t_i^n, \, m \to n$, получаем парадокс, т.е. говорят, что процесс приславший сообщение \elki{из прошлого} отстает во времени~\cite{okol}.

Для того, чтобы парадоксов времени в системе не возникало, необходимо предусмотреть специальные программы, которые бы синхронизировали по времени процессы в моделируемой системе. Такие программы получили название алгоритмов синхронизации модельного времени. Алгоритмы синхронизации разнообразны, но все их как правило можно разделить на два основных класса: консервативные и оптимистические.

Если синхронизация модельного времени процессов происходит под управлением консервативного алгоритма, это предполагает, что процесс получает сообщения в том же порядке, в котором их посылает ему отправитель.


Существует несколько технологий на уровне архитектур, которые позволяют задать основные правила создания распределенных имитационных моделей: DDS, ARIS, HLA и др.

\todo{Раскрыть HLA, т.е. почему занимаюсь и рассматриваю архитектуру ХЛА, краткое ее описание, основных достоинств и недостатков}

DDS (Data Distribution Service) \nom[]{DDS}{Data Distribution Service} --- спецификация, описывающая датацентрическую модель, использующая шаблон обмена сообщениями по технологии <<издатель-подписчик>>. Эта спецификация определяет как интерфейсы приложений (API)\nom[]{API}{Application Interfaces} и cемантика доставляют информацию от издателя к соответствующим потребителям.

ARIS (Architecture of Integrated Information Systems) \nom[]{ARIS}{Architecture of Integrated Information Systems} ---  методология и тиражируемый программный продукт для моделирования бизнес-процессов организаций. Общий принцип --- возможность интеграции моделей разных типов в рамках одного репозитория посредством декомпозиции (детализации) объектов. Таким образом, любую организацию можно описать с помощью иерархии моделей --- от обобщения: например, процессы верхнего уровня до уровня процедур и ресурсного окружения функций.

HLA (High-level architecture) --- архитектура высокого уровня, представляет собой архитектуру общего назначения для распределенных компьютерных имитационных систем. Является стандартом IEEE~1516.
В соответствии с правилами построения имитационных моделей по архитектуре HLA были реализованы, например, такие проекты как объединение центров управления космическими аппаратами в рамках программы МКС в единую распределенную систему (РКК Энергия/Королев --- ATVCC/Тулуза --- NASA JSC/Хьюстон); предшественник HLA --- сеть SIMNET реально использовалась для тренировок и поддержки военных операций (а также их последующего анализа), например, в 1992~году в рамках кампании <<Буря в пустыне>>; 333-й Центр боевой подготовки сухопутных войск в п.\,Мулино и др.

В имитационном моделировании принято различать три понятия времени: физическое, модельное, процессорное \cite{okol}. Сущностью имитационного моделирования является продвижение модельного времени при выполнении модели и выполнение событий, связанных с определенными значениями модельного времени. В распределенной модели первичной единицей является логический процесс. Каждый логический процесс выполняется в своем модельном времени как самостоятельная последовательная модель. Логический процесс взаимодействует с другими процессами, передавая им сообщения. При распределенной реализации модельное время в разных логических процессах движется с разными скоростями и в некоторый произвольный момент времени оказывается разным.

\todo{Описать подробно алгоритмы консервативные и оптимистические. Какие бывают разновидности в чем их суть и смысл. }

\subsection{Консервативные алгоритмы синхронизации} \label{subsect1}

Предположим, что процессы выполняются на распределенной системе. И логический процесс $B$ в силу определенных причин (например, резко возросла нагрузка на процессор) выполняется медленнее. Следовательно возникли задержки. Для логического процесса $W$ получение сообщения в момент времени $t5$ означает изменение прошлого в момент времени $t4$, что нарушает корректную последовательность развития моделирования, рисунок~\ref{fig:Выполнение модели <<Ватерлоо>> под управлением консервативного алгоритма синхронизации}.

\begin{figure}[!ht]
\centering
\includegraphics[scale=1]{images/waterloo-kas.pdf}
\caption{Выполнение модели <<Ватерлоо>> под управлением консервативного алгоритма синхронизации}
\label{fig:Выполнение модели <<Ватерлоо>> под управлением консервативного алгоритма синхронизации}
\end{figure}

Иначе, консервативные алгоритмы блокируют продвижение модельного времени в том случае, если процесс получил сообщение от процесса с меньшим временем.


\subsection{Оптимистические алгоритмы синхронизации} \label{subsect2}

Оптимистические алгоритмы в аналогичном случае осуществляют откат времени процесса с большим временем до времени процесса, приславшего сообщение, обрабатывает его, а так же заново обрабатывают все сообщения от этого времени до текущего в правильной временной последовательности.

\begin{figure}[!ht]
\centering
\includegraphics[scale=1]{images/waterloo-oas.pdf}
\caption{Выполнение модели <<Ватерлоо>> под управлением оптимистического алгоритма синхронизации}
\label{fig:Выполнение модели <<Ватерлоо>> под управлением оптимистического алгоритма синхронизации}
\end{figure}

%\newpage
%============================================================================================================================

\section{Ссылки} \label{sect1_2}
Сошлёмся на библиографию. Одна ссылка: \cite[с.~54]{Sokolov}\cite[с.~36]{Gaidaenko}. Две ссылки: \cite{Sokolov,Gaidaenko}. Много ссылок:  \cite[с.~54]{Lermontov,Management,Borozda} \cite{Lermontov,Management,Borozda,Marketing,Constitution,FamilyCode,Gost.7.0.53,Razumovski,Lagkueva,Pokrovski,Sirotko,Lukina,Methodology,Encyclopedia,Nasirova,Berestova,Kriger}. И ещё немного ссылок: \cite{Article,Book,Booklet,Conference,Inbook,Incollection,Manual,Mastersthesis,Misc,Phdthesis,Proceedings,Techreport,Unpublished}. \cite{medvedev2006jelektronnye, CEAT:CEAT581, doi:10.1080/01932691.2010.513279,Gosele1999161,Li2007StressAnalysis, Shoji199895,test:eisner-sample,AB_patent_Pomerantz_1968,iofis_patent1960}

%Попытка реализовать несколько ссылок на конкретные страницы для стандартной реализации:[\citenum{Sokolov}, с.~54; \citenum{Gaidaenko}, с.~36].

%Несколько источников мультицитата \cites[vii--x, 5, 7]{Sokolov}[v--x, 25, 526]{Gaidaenko} поехали дальше

Ссылки \nom[]{ОЛОЛ}{Это ссылки конечно же.} на собственные работы:~\cite{vakbib1, confbib1}

Сошлёмся на приложения: Приложение \ref{AppendixA}, Приложение \ref{AppendixB2}.

Сошлёмся на формулу: формула \eqref{eq:equation1}.

Сошлёмся на изображение: рисунок \ref{img:knuth}.
Имитационное моделирование систем --- искусство и наука. \cite{Shennon}

%\newpage
%============================================================================================================================

\section{Формулы} \label{sect1_3}

Благодаря пакету \textit{icomma}, \LaTeX~одинаково хорошо воспринимает в качестве десятичного разделителя и запятую ($3,1415$), и точку ($3.1415$).

\subsection{Ненумерованные одиночные формулы} \label{subsect1_3_1}

Вот так может выглядеть формула \nom[]{НИИИ}{Формула такая формула, ололо}, которую необходимо вставить в строку по тексту: $x \approx \sin x$ при $x \to 0$.

А вот так выглядит ненумерованая отдельностоящая формула \Gls{math} c подстрочными и надстрочными индексами:
\[
(x_1+x_2)^2 = x_1^2 + 2 x_1 x_2 + x_2^2
\]

При использовании дробей формулы могут получаться очень высокие:
\[
  \frac{1}{\sqrt{2}+
  \displaystyle\frac{1}{\sqrt{2}+
  \displaystyle\frac{1}{\sqrt{2}+\cdots}}}
\]

В формулах можно использовать греческие буквы:
\[
\alpha\beta\gamma\delta\epsilon\varepsilon\zeta\eta\theta\vartheta\iota\kappa\lambda\\mu\nu\xi\pi\varpi\rho\varrho\sigma\varsigma\tau\upsilon\phi\varphi\chi\psi\omega\Gamma\Delta\Theta\Lambda\Xi\Pi\Sigma\Upsilon\Phi\Psi\Omega
\]

\def\slantfrac#1#2{ \hspace{3pt}\!^{#1}\!\!\hspace{1pt}/
  \hspace{2pt}\!\!_{#2}\!\hspace{3pt}
} %Макрос для красивых дробей в строчку (например, 1/2)
Для красивых дробей (например, в индексах) можно добавить макрос
\verb+\slantfrac+ и писать $\slantfrac{1}{2}$ вместо $1/2$.
%\newpage
%============================================================================================================================

\subsection{Ненумерованные многострочные формулы} \label{subsect1_3_2}

Вот так можно написать две формулы, не нумеруя их, чтобы знаки равно были строго друг под другом:
\begin{align}
  f_W & =  \min \left( 1, \max \left( 0, \frac{W_{soil} / W_{max}}{W_{crit}} \right)  \right), \nonumber \\
  f_T & =  \min \left( 1, \max \left( 0, \frac{T_s / T_{melt}}{T_{crit}} \right)  \right), \nonumber
\end{align}

Выровнять систему ещё и по переменной $ x $ можно, используя окружение \verb|alignedat| из пакета \verb|amsmath|. Вот так: 
\[
    |x| = \left\{
    \begin{alignedat}{2}
        &&x, \quad &\text{eсли } x\geqslant 0 \\
        &-&x, \quad & \text{eсли } x<0
    \end{alignedat}
    \right.
\]
Здесь первый амперсанд  означает выравнивание по~левому краю, второй "--- по~$ x $, а~третий "--- по~слову <<если>>. Команда \verb|\quad| делает большой горизонтальный пробел. 

Ещё вариант:
\[
    |x|=
    \begin{cases}
    \phantom{-}x, \text{если } x \geqslant 0 \\
    -x, \text{если } x<0
    \end{cases}
\]

Кроме того, для  нумерованых формул \verb|alignedat|  делает вертикальное
выравнивание номера формулы по центру формулы. Например,  выравнивание компонент вектора:
\begin{equation}
 \label{eq:2p3}
 \begin{alignedat}{2}
{\mathbf{N}}_{o1n}^{(j)} = \,{\sin} \phi\,n\!\left(n+1\right)
         {\sin}\theta\,
         \pi_n\!\left({\cos} \theta\right)
         \frac{
               z_n^{(j)}\!\left( \rho \right)
              }{\rho}\,
           &{\boldsymbol{\hat{\mathrm e}}}_{r}\,+   \\
+\,
{\sin} \phi\,
         \tau_n\!\left({\cos} \theta\right)
         \frac{
            \left[\rho z_n^{(j)}\!\left( \rho \right)\right]^{\prime}
              }{\rho}\,
            &{\boldsymbol{\hat{\mathrm e}}}_{\theta}\,+   \\
+\,
{\cos} \phi\,
         \pi_n\!\left({\cos} \theta\right)
         \frac{
            \left[\rho z_n^{(j)}\!\left( \rho \right)\right]^{\prime}
              }{\rho}\,
            &{\boldsymbol{\hat{\mathrm e}}}_{\phi}\:.
\end{alignedat}
\end{equation}

Ещё об отступах. Иногда для лучшей "читаемости" формул полезно
немного исправить стандартные интервалы \LaTeX с учётом логической
структуры самой формулы. Например в формуле~\ref{eq:2p3} добавлен
небольшой отступ \verb+\,+ между основными сомножителями, ниже
результат применения всех вариантов отступа:
\begin{align*}
\backslash! &\quad f(x) = x^2\! +3x\! +2 \\
  \mbox{по-умолчанию} &\quad f(x) = x^2+3x+2 \\
\backslash, &\quad f(x) = x^2\, +3x\, +2 \\
\backslash{:} &\quad f(x) = x^2\: +3x\: +2 \\
\backslash; &\quad f(x) = x^2\; +3x\; +2 \\
\backslash \mbox{space} &\quad f(x) = x^2\ +3x\ +2 \\
\backslash \mbox{quad} &\quad f(x) = x^2\quad +3x\quad +2 \\
\backslash \mbox{qquad} &\quad f(x) = x^2\qquad +3x\qquad +2
\end{align*}


Можно использовать разные математические алфавиты:
\begin{align}
\mathcal{ABCDEFGHIJKLMNOPQRSTUVWXYZ} \nonumber \\
\mathfrak{ABCDEFGHIJKLMNOPQRSTUVWXYZ} \nonumber \\
\mathbb{ABCDEFGHIJKLMNOPQRSTUVWXYZ} \nonumber
\end{align}

Посмотрим на систему уравнений на примере аттрактора Лоренца:

\[ 
\left\{
  \begin{array}{rl}
    \dot x = & \sigma (y-x) \\
    \dot y = & x (r - z) - y \\
    \dot z = & xy - bz
  \end{array}
\right.
\]

А для вёрстки матриц удобно использовать многоточия:
\[ 
\left(
  \begin{array}{ccc}
  	a_{11} & \ldots & a_{1n} \\
  	\vdots & \ddots & \vdots \\
  	a_{n1} & \ldots & a_{nn} \\
  \end{array}
\right)
\]


%\newpage
%============================================================================================================================
\subsection{Нумерованные формулы} \label{subsect1_3_3}

А вот так пишется нумерованая формула:
\begin{equation}
  \label{eq:equation1}
  e = \lim_{n \to \infty} \left( 1+\frac{1}{n} \right) ^n
\end{equation}

Нумерованых формул может быть несколько:
\begin{equation}
  \label{eq:equation2}
  \lim_{n \to \infty} \sum_{k=1}^n \frac{1}{k^2} = \frac{\pi^2}{6}
\end{equation}

Впоследствии на формулы (\ref{eq:equation1}) и (\ref{eq:equation2}) можно ссылаться.

Сделать так, чтобы номер формулы стоял напротив средней строки, можно, используя окружение \verb|multlined| (пакет \verb|mathtools|) вместо \verb|multline| внутри окружения \verb|equation|. Вот так:
\begin{equation} % \tag{S} % tag - вписывает свой текст 
  \label{eq:equation3}
    \begin{multlined}
        1+ 2+3+4+5+6+7+\dots + \\ 
        + 50+51+52+53+54+55+56+57 + \dots + \\ 
        + 96+97+98+99+100=5050 
    \end{multlined}
\end{equation}

Используя команду \verb|\labelcref| из пакета \verb|cleveref|, можно
красиво ссылаться сразу на несколько формул
(\labelcref{eq:equation1,eq:equation3,eq:equation2}), даже перепутав
порядок ссылок \verb|(\labelcref{eq:equation1,eq:equation3,eq:equation2})|.

