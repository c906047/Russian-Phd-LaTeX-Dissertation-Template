%\chapter*{Словарь терминов}             % Заголовок
\printglossary[title=Словарь терминов, toctitle=Словарь терминов]
\addcontentsline{toc}{chapter}{Словарь терминов}  % Добавляем его в оглавление

%\textbf{\TeX} - Cистема компьютерной вёрстки, разработанная американским профессором информатики Дональдом Кнутом
%
%\textbf{Панграмма} - Короткий текст, использующий все или почти все буквы алфавита
