\documentclass[a4paper,14pt]{extreport}
\usepackage{polyglossia}
\setdefaultlanguage{russian}
\setotherlanguages{english,german}
\usepackage[labelsep=period, nooneline]{caption} % Заменить умолчальное разделение ':' на '.'в подписях к рисункам и таблицам
\defaultfontfeatures{Mapping=tex-text}% Для работы кавычек, тире и т.д.
%\setmainfont{Times New Roman} % Основной шрифт документа
%\setmainfont{CharterITC}
\setmainfont{CMU Serif Roman}
\newfontfamily{\cyrillicfonttt}{CMU Serif Roman}

%Поиск в пдф-документе
\usepackage{cmap}

% Пакеты для xelatex
\usepackage{xunicode, xltxtra}

% Работа с геометрией страницы
\usepackage{geometry}

% Дополнительная работа с математикой
\usepackage{mathtext} % Русские буквы в формулах
\usepackage{amsmath, amssymb, amsthm, amsfonts, mathtools} % AMS
\usepackage{icomma} % "Умная" запятая: $0,2$ --- число, $0, 2$ --- перечисление

% Шрифты
\usepackage{euscript} % Шрифт Евклид
\usepackage{mathrsfs} % Красивый матшрифт

% Пакеты для работы с таблицами
\usepackage{verbatim, array, booktabs, multirow, makecell, tabu, longtable, xcolor, colortbl, wrapfig}

% Для печати в несколько колонок
\usepackage{multicol}

% Для двойной горизонтальной линии в таблицах
\usepackage{hhline}

% Для подключения eps графики
\usepackage{epstopdf}

% Для пробелов в названиях и путях к картинкам
\usepackage{graphicx}
\usepackage[space]{grffile}

% Неразобранные, но нужные пакеты (подключались по мере необходимости для выполнения разных задач)
\usepackage{titletoc, rotating, fp, fontspec, setspace, cite, tikz, tocloft, ifpdf, kvoptions, subfig, latexsym, listings, pgfplots, pgfpages, xifthen, ulem, float, indentfirst, enumerate, titlesec, enumitem, todonotes, lettrine, lipsum, pdflscape, pdfpages, ragged2e, lastpage, etoolbox}

% Библиотеки для графического пакета tikz
\usetikzlibrary{intersections, fit, automata, calc, trees, positioning, arrows, chains, shapes.geometric, decorations.pathreplacing, decorations.pathmorphing, shapes, matrix, shapes.symbols, shadows, backgrounds, mindmap, shapes.gates.logic.US, trees, positioning, circuits.logic.IEC, shapes.gates.logic.IEC, shapes.geometric
}

% Подключаем графический пакет для рисования диаграмм
\usepgfmodule{plot}

% Задаем геометрию полей страницы
\geometry{left=2cm}
\geometry{right=1cm}
\geometry{top=2cm}
\geometry{bottom=2cm}

% Заменяем вводимые с клавиатуры кавычки на елочки
\usepackage{csquotes}
%\MakeOuterQuote{"}

% Выводит в Содержании section ЗАГЛАВНЫМИ буквами
%\usepackage{regexpatch}
%\makeatletter
%\xpatchcmd*{\@sect}{\fi#7}{\fi\@nameuse{format#1}{#7}}{}{}
%
%%%% for sections and subsections we want uppercase
%\protected\def\formatsection{\MakeUppercase}
%\protected\def\formatsubsection{\MakeUppercase}
%
%%%% the other titles are left unchanged
%\let\formatsubsubsection\@firstofone
%\let\formatparagraph\@firstofone
%\let\formatsubparagraph\@firstofone
%
%%%% the following is necessary only if hyperref is used
%\AtBeginDocument{%
%  \pdfstringdefDisableCommands{%
%    \let\formatsection\@firstofone
%    \let\formatsubsection\@firstofone
%  }%
%}
%\makeatother



% Настройка Содержания
\gappto\captionsrussian{\renewcommand{\contentsname}{Оглавление}}
\renewcommand{\cfttoctitlefont}{\hspace{0.38\textwidth}\MakeUppercase
}
\renewcommand{\cftbeforetoctitleskip}{-2em}% Отступ от верхнего края до Содержания
\renewcommand{\cftaftertoctitle}{\mbox{}\hfill \\ \mbox{}\hfill{}\vspace{-2.5em}}% Отступ от Содержания до текста

% Запрещаем переносы в Содержании
\makeatletter
\renewcommand{\@tocrmarg}{2.55em plus1fil}
\makeatother
\setcounter{secnumdepth}{3} % глубина глав 2 (чаптер, секшн, субсекшн)
\setcounter{tocdepth}{3} % глубина оглавления

% Обозначения и сокращения
\usepackage[russian]{nomencl}
\newcommand{\nomdescr}[1]{{--\hskip0.7cm}\parbox[t]{15cm}{\RaggedRight #1}}
\newcommand{\nomwithdim}[3]{\nomenclature[#1]{#2}{\nomdescr{#3}}}
\newcommand{\nom}[3][]{\nomwithdim{R#1}{#2}{#3}}
\makenomenclature
\renewcommand{\nomname}{%
\vspace{-3.5cm}
\centerline{%
\normalfont
\normalsize
\MakeUppercase{\textmd{Список сокращений и условных обозначений}}
}
\vspace{-0.5cm}
}
\setlength{\nomitemsep}{-0.25cm}

% Определяем кавычки <<елочки>>
\DeclareQuoteStyle{russian}
{\guillemotleft}{\guillemotright}[0.025em]
{\quotedblbase}{\textquotedblleft}
\ExecuteQuoteOptions{style=russian}
\newcommand{\elki}[1]{\enquote{#1}}

% Списки --, 1. и а) с отступом первой строки
% Нумерация + вложенная нумерация
\renewcommand{\alph}[1]{\asbuk{#1}} 
\setlist{nolistsep}
\setitemize[1]{label=--, fullwidth, itemindent=\parindent, listparindent=\parindent}
\setitemize[2]{label=--, fullwidth, itemindent=\parindent, listparindent=\parindent, leftmargin=\parindent}
\setitemize[3]{label=--, fullwidth, itemindent=\parindent, listparindent=\parindent, leftmargin=\parindent}
\setenumerate[1]{label=\arabic*., fullwidth, itemindent=\parindent, listparindent=\parindent}
\setenumerate[2]{label=\arabic{enumi}.\arabic{enumii}, fullwidth, itemindent=\parindent, listparindent=\parindent, leftmargin=\parindent}
\setenumerate[3]{label=\arabic{enumi}.\arabic{enumii}.\arabic{enumiii}, fullwidth, itemindent=\parindent, listparindent=\parindent, leftmargin=\parindent}
\setenumerate[4]{label=\alph*), fullwidth, itemindent=\parindent, 
listparindent=\parindent, leftmargin=\parindent}

% Формат подрисуночной (табличной, формульной) надписи и их нумерация
\makeatletter
\@addtoreset{table}{section}
\@addtoreset{figure}{section}
\makeatother
\addto\captionsrussian{\def\figurename{Рисунок}
\renewcommand{\tablename}{Таблица}}
\renewcommand{\thefigure}{\thesection.\arabic{figure}}
\renewcommand{\thetable}{\thesection.\arabic{table}}
\renewcommand{\theequation}{\thesection.\arabic{equation}}
%\setlength{\intextsep}{5mm}%Увеличиваем расстояние между основным текстом и рисунком под ним
%\setlength\belowcaptionskip{-5mm}%Уменьшаем расстояние между основным текстом и рисунком над ним
\RequirePackage{caption}
\DeclareCaptionLabelFormat{fullparens}{\bothIfFirst{#1}{~}#2}
\captionsetup[figure]{
format=plain,
indention=0.7cm,
justification=centering,
labelsep=endash,
skip=5mm
}
\captionsetup[table]{
format=plain,
font={footnotesize},
labelformat=fullparens,
labelsep=endash,
%labelfont=it,
%textfont=bf,
justification=raggedright,
singlelinecheck=false
}
\makeatletter
\setlength\abovecaptionskip{2\p@}
\setlength\belowcaptionskip{1\p@}
\makeatother

% Переопределение списков itemize и enumerate, убирающее расстояния между строками и абзацами
\makeatletter
\renewcommand{\@listI}{%
\topsep=0pt }
\makeatother

\makeatletter
\let\old@itemize=\itemize
\def\itemize{\old@itemize
\setlength{\itemsep}{0pt}
\setlength{\parskip}{0pt}
\setlength{\leftskip}{0pt}
}
\makeatother

\makeatletter
\let\old@enumerate=\enumerate
\def\enumerate{\old@enumerate
\setlength{\itemsep}{0pt}
\setlength{\parskip}{0pt}
\setlength{\leftskip}{0pt}
}
\makeatother


%%% Определение subsubsubsection
\makeatletter
\newcommand{\subsubsubsection}[1]{\refstepcounter{subsubsubsection}
        \vspace{15pt plus 7pt minus 3pt}\nobreak\par%
        {\thesubsubsubsection\space #1}%
        \addcontentsline{toc}{subsubsubsection}{\thesubsubsubsection\space #1}%
        \par\vspace{2ex}}
\newcounter {subsubsubsection}[subsubsection]

\renewcommand\thesubsubsubsection{\thesubsubsection .\@arabic\c@subsubsubsection}
 \makeatother
 
 
% Названия глав располагаем с красной строки и устанавливаем их размер и отступы до и после названия
\makeatletter
%\renewcommand\section{
%\@startsection {section}{1}{\parindent}{\z@}%
%{2ex \@plus.2ex}%
%{\hyphenpenalty=10000\normalfont\normalsize
%\vspace{4ex}
%\MakeUppercase}
%}
\def\section{\@dblarg\@section}
\def\@section[#1]#2{\refstepcounter{section}%
        {\newpage\nobreak\hspace*{.5cm}\raggedright
        \llap{\thesection\space\space}\uppercase{#2}}%
        \addcontentsline{toc}{section}{\thesection\space #1}\par\nobreak}
\renewcommand{\subsection}[1]{\refstepcounter{subsection}
        \vspace{15pt plus 7pt minus 3pt}\nobreak\par%
        {\thesubsection\space #1}%
        \addcontentsline{toc}{subsection}{\thesubsection\space #1}%
        \par\vspace{2ex}}
\renewcommand{\subsubsection}[1]{\refstepcounter{subsubsection}
        \vspace{15pt plus 7pt minus 3pt}\nobreak\par%
        {\thesubsubsection\space #1}%
        \addcontentsline{toc}{subsubsection}{\thesubsubsection\space #1}%
        \par\vspace{2ex}}

\makeatother

% Убираем точки после цифры в главах и подглавах
\makeatletter
\renewcommand\thechapter{\@arabic\c@chapter}
\renewcommand\thesection{\@arabic\c@section}
\gappto\russian@capsformat{\renewcommand{\postsection}{\@aftersepkern}}
\renewcommand\thesubsection{\@arabic\c@section.\@arabic\c@subsection}
\gappto\russian@capsformat{\renewcommand{\postsubsection}{\@aftersepkern}}
\renewcommand\thesubsubsection{\@arabic\c@section.\@arabic\c@subsection.\@arabic\c@subsubsection}
\gappto\russian@capsformat{\renewcommand{\postsubsubsection}{\@aftersepkern}}
\makeatother


% Задаем отступ от нумерации названия глав и подглав в Содержании
\makeatletter
\newcommand{\l@likechapter}{\@dottedtocline{1}{1em}{1em}}
\renewcommand{\l@chapter}{\@dottedtocline{1}{1em}{1em}}
\renewcommand*\l@section{\@dottedtocline{1}{1em}{1em}}%
\renewcommand*\l@subsection{\@dottedtocline{1}{2.0em}{1.7em}}%
\renewcommand*\l@subsubsection{\@dottedtocline{1}{3.6em}{2.4em}}%
\newcommand*\l@subsubsubsection{\@dottedtocline{1}{6em}{2.4em}}%
\makeatother


% Работа с датой
\usepackage[ddmmyyyy]{datetime}
\renewcommand{\dateseparator}{.}
\def\daterussian{%
\def\mon{\ifcase\month\or января\or февраля\or марта 
\or апреля\or мая\or июня\or июля\or августа\or сентября 
\or октября\or ноября\or декабря\fi\space}%
}



% Создание счетчиков страниц, разделов, рисунков, таблиц, приложений
\usepackage[square,numbers,sort&compress]{natbib}

\newcounter{reference}
\pretocmd{\bibitem}{\addtocounter{reference}{1}}{}{}

\usepackage[chapter,     %
            page,         %
            section,     %
            figure,      %
            table,       %
            reference,   %
            ]{totalcount}

% Оставил этот пример. Из него можно вытянуть дополнительные счетчики, если они понадобятся (например можно сделать счетчик чего-то в документе отдельно от приложений)
%\newcounter{totfigures}
%\newcounter{tottables}
%\newcounter{totsections}
%\newcounter{totreferences}
%\newcounter{totchapter}
%\makeatletter
%\AtEndDocument{%
%\addtocounter{totsections}{\value{section}}%
%\addtocounter{totfigures}{\value{figure}}%
%\addtocounter{tottables}{\value{table}}%
%\addtocounter{totchapter}{\value{chapter}}%
%\immediate\write\@mainaux{%
%\string\gdef\string\totsec{\number\value{totsections}}%
%\string\gdef\string\totfig{\number\value{totfigures}}%
%\string\gdef\string\tottab{\number\value{tottables}}%
%\string\gdef\string\totref{\number\value{totreferences}}%
%\string\gdef\string\totpril{\number\value{totchapter}}%
%}%
%}
%\makeatother
%\pretocmd{\section}{\addtocounter{totfigures}{\value{figure}}}{}{}
%\pretocmd{\section}{\addtocounter{tottables}{\value{table}}}{}{}
%\pretocmd{\chapter}{\addtocounter{totfigures}{\value{figure}}}{}{}
%\pretocmd{\chapter}{\addtocounter{tottables}{\value{table}}}{}{}
%\pretocmd{\bibitem}{\addtocounter{totreferences}{1}}{}{}
%\pretocmd{\chapter}{\addtocounter{totchapter}{0}}{}{}
%\pretocmd{\section}{\addtocounter{totsections}{1}}{}{}% При включении Приложения раскомментировать


% Создание своего оформления для произвольной главы
\newcommand{\empline}{\mbox{}\newline}
\newcommand{\likechapter}[1]{
\newpage
\begin{center}
\MakeUppercase{#1}
\end{center}
\vspace{-1cm}
\empline}



% Настройка Приложений
\usepackage[title,titletoc]{appendix}
%\newcommand{\empline}{\mbox{}\newline} % пустая строка
\newcommand{\append}[1]{%
    \clearpage
    \refstepcounter{chapter}
    \begin{center}
       \MakeUppercase{%
       \appendixname~\Asbuk{chapter}}
    \end{center}
    \begin{center}{#1}\end{center}
    \renewcommand{\thefigure}{\Asbuk{chapter}.\arabic{figure}}
	\renewcommand{\thetable}{\Asbuk{chapter}.\arabic{table}}
	\renewcommand{\theequation}{\Asbuk{chapter}.\arabic{equation}}
	\renewcommand{\thesubsection}{\Asbuk{chapter}.\arabic{subsection}}
	\renewcommand{\thesubsubsection}{\Asbuk{chapter}.\arabic{subsection}.\arabic{subsubsection}}
	\renewcommand{\thesubsubsubsection}{\Asbuk{chapter}.\arabic{subsection}.\arabic{subsubsection}.\arabic{subsubsubsection}}
    \empline
    %В приложениях эта строка вставлена, поэтому здесь коментим
    %\addcontentsline{toc}{chapter}{\Asbuk{chapter}\hspace{0em}~#1}
    }


% Создание примечания
\newcommand{\prim}{\todo[color=green!40,inline,caption={\textit{Примечание}}]}

%Переносы вкл. и 1.5 интервала выкл.
\newcommand{\pereodin}{\hyphenpenalty=100\singlespacing}

%%% Настройка библиографии
\makeatletter
%Стиль библиографических ссылок БибТеХа - нумеровать в порядке упоминания в тексте
\bibliographystyle{%
%ugost2008%
../BibTeX-Styles/ugost2008%
%BibTeX-Styles/utf8gost71u%
}
%Заменяем квадратные скобки в списке литературы на цифру без точки, если нужна точка, то ставим ее после {#1.}
\renewcommand{\@biblabel}[1]{#1}
\makeatother
\newcommand\mybibname{Список использованных источников}
% bibsection используется в пакете natbib
\renewcommand\bibsection{%
\newpage\centering\vskip2cm
\noindent\normalsize\MakeUppercase{\mybibname}%
\markboth{\MakeUppercase{\mybibname}}{\MakeUppercase{\mybibname}}%
\addcontentsline{toc}{section}{Список использованных источников}
\vskip1cm}

\sloppy%Перенос слов, которые не убираются в строке (вылезают на поля)
\clubpenalty=10000 % Запрещаем разрыв страницы после первой строки абзаца
\widowpenalty=10000 % Запрещаем разрыв страницы после последней строки абзаца
\brokenpenalty=10000 % Подавление переносов на другую страницу
\hyphenpenalty=10000 % Запрещаем переносы
\parfillskip=0pt plus .95\textwidth % Последняя строка абзаца не будет слишком короткой
\parindent=1cm% Абзацный отступ
%\renewcommand{\baselinestretch}{1.3}% Межстрочный интервал
\onehalfspacing
\renewcommand{\ge}{\geqslant}
\renewcommand{\le}{\leqslant}
\righthyphenmin=2 % Минимальное число символов при переносе - 2.
\tabulinesep=1mm%Расстояние между нижней чертой и содержимым в ячейке таблицы
\newcommand\marker[2]{{\fboxsep=0pt\colorbox{green!50}{\strut }}}
\usepackage{hyperref}
\usepackage{hyperxmp}
\hypersetup
{
    pdfauthor={User},
    pdfcreator={Microsoft® Word 2010},
    pdfproducer={Microsoft® Word 2010}
}
\usepackage{fancyhdr}

%%% Переопределяем шрифт в заголовках таблиц
%\renewcommand\theadset{\def\arraystretch{.75}}
\renewcommand\theadfont{\bfseries\footnotesize}
%%% Уменьшаем вертикальные расстояния в заголовках таблиц сверху и снизу от подписи
\renewcommand\theadgape{\Gape[0pt][0pt]}

%%% Временно
\makeatletter
\newcommand{\verbatimfont}[1]{\renewcommand{\verbatim@font}{\ttfamily#1}}
\makeatother

%%%%%%%%%%%%%%%%%%%%%%%%%%%%%%%%%