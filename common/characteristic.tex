
{\actuality} 

\todo{Что во введении? Переписать скорее всего.}

Благодаря интенсивному развитию вычислительной техники моделирование приобретает общенаучный характер и применяется в исследованиях объектов и процессов, происходящих в природе, в науках о человеке и обществе. Зачастую моделирование используется в тех случаях, когда исследование процесса невозможно или невыгодно по определенным причинам, например, это может быть высокая стоимость создания процесса, его сложное строение, небольшие пространственно-временные размеры и т.д.

Особую популярность в последние годы приобрело имитационное моделирование --- моделирование, при котором моделирующий алгоритм с той или иной степень точности воспроизводит функционирование исходной системы. Имитационная модель воспроизводит поведение моделируемой системы во времени~\cite{disksobmod}.

В имитационном моделировании принято различать три понятия времени: физическое, модельное, процессорное~\cite{voz-disser}. Сущностью имитационного моделирования является продвижение модельного времени при выполнении модели и выполнение событий, связанных с определенными значениями модельного времени.

Имитационная модель может быть выполнена на распределенной вычислительной системе, где выигрыш по времени выполнения модели достигается за счет параллельного выполнения событий, запланированных на разные моменты модельного времени. В распределенной модели первичной единицей является логический процесс. Каждый логический процесс выполняется в своем модельном времени как самостоятельная последовательная модель. Логический процесс взаимодействует с другими процессами, передавая им сообщения. При распределенной реализации модельное время в разных логических процессах движется с разными скоростями и в некоторый произвольный момент времени оказывается разным. Для того, чтобы модель правильно воспроизводила последовательность событий в моделируемой системе, необходимо чтобы не возникало парадоксов времени. Тот процесс продвигает свое время вперед, который получил сообщение от процесса с большим временем. Если процесс получает сообщение от процесса с меньшим временем, получаем парадокс, т.е. говорят, что процесс приславший сообщение \elki{из прошлого} отстает во времени.

Для того, чтобы парадоксов времени в системе не возникало, необходимо предусмотреть специальные программы, которые бы синхронизировали по времени процессы в моделируемой системе. Такие программы получили название алгоритмов синхронизации модельного времени. Алгоритмы синхронизации разнообразны, но все их как правило можно разделить на два основных класса: консервативные и оптимистические.

Если синхронизация модельного времени процессов происходит под управлением консервативного алгоритма, это предполагает, что процесс получает сообщения в том же порядке, в котором их посылает ему отправитель.
Иначе, консервативные алгоритмы блокируют продвижение модельного времени в том случае, если процесс получил сообщение от процесса с меньшим временем.

Оптимистические алгоритмы в аналогичном случае осуществляют откат времени процесса с большим временем до времени процесса, приславшего сообщение, обрабатывает его, а так же заново обрабатывают все сообщения от этого времени до текущего в правильной временной последовательности.


%Обзор, введение в тему, обозначение места данной работы в
%мировых исследованиях и~т.\:п., можно использовать ссылки на другие
%работы~\cite{Gosele1999161} (если их нет, то в автореферате
%автоматически пропадёт раздел <<Список литературы>>). Внимание! Ссылки
%на другие работы в разделе общей характеристики работы можно
%использовать только при использовании \verb!biblatex! (из-за технических
%ограничений \verb!bibtex8!. Это связано с тем, что одна и та же
%характеристика используются и в тексте диссертации, и в
%автореферате. В последнем, согласно ГОСТ, должен присутствовать список
%работ автора по теме диссертации, а \verb!bibtex8! не умеет выводить в одном
%файле два списка литературы).


{\aim} данной работы является разработка \todo{адаптивного} алгоритма синхронизации и соответствующего ему метода продвижения времени. Суть работы адаптивного алгоритма синхронизации заключается в анализе продвижения времени каждым процессом, выявлении признаков, характеризующих тип процесса и переключении продвижения времени этим процессом на наиболее оптимальный для данного типа процесса алгоритм продвижения модельного времени. Подобный подход позволяет снизить количество простоев и откатов процессов, возникающие в консервативном и классическом оптимистическом алгоритмах синхронизации, что влечет за собой уменьшение временных потерь на моделирование.


Для~достижения поставленной цели необходимо было решить следующие {\tasks}:
\begin{enumerate}
 % \item Исследовать, разработать, вычислить и~т.\:д. и~т.\:п.
    \item Исследовать существующие алгоритмы синхронизации времени, построить их модели, оценить их характеристики, эффективность, выявить их недостатки.
  \item Разработать динамический алгоритм оценки характеристик локальных процессов.
  \item Разработать адаптивный алгоритм синхронизации, переключающий тип синхронизации в соответствии с типом процесса, продвигающего свое модельное время.
  \item Вычислить эффективность адаптивного алгоритма синхронизации в сравнении с классическими алгоритмами.
  \item Разработать предложения по применению адаптивного алгоритма синхронизации времени в распределенных системах имитационного моделирования.
\end{enumerate}


{\novelty}
\begin{enumerate}
  \item Впервые разработан динамический алгоритм анализа характеристик локальных процессов.
  \item Впервые разработан адаптивный алгоритм продвижения времени, переключающий тип синхронизации в соответствии с типом процесса, продвигающего свое время.
  %\item Было выполнено оригинальное исследование \ldots
\end{enumerate}

{\influence} определяется \ldots Разработанные теоретические положения отражены в заявке на изобретение~№

{\methods} Научной основой для решения поставленной
задачи являются: методы системного анализа и моделирования, вычислительных экспериментов, анализа временных рядов, теория вероятностей и математической статистики, теория эффективности \todo{целенаправленных} процессов.

{\defpositions}
\begin{enumerate}
  \item Динамический алгоритм анализа характеристик локальных процессов, позволяющий на основании изменения приращения их локального времени, а так же интенсивности посылки этими процессами сообщений другим процессам, участвующим в моделировании определить их принадлежность к определенному типу имитационных моделей.
  \item Адаптивный алгоритм синхронизации времени, позволяющий в зависимости от типа локального процесса переключать алгоритм синхронизации времени на оптимальный для данного типа, что позволяет уменьшить количество простоев в сравнении с применением только консервативного алгоритма синхронизации и откатов в сравнении с применением только оптимистического алгоритма синхронизации и повышает эффективность моделирования за счет выполнения имитационной модели за более короткое модельное время.
  \item \todo{Третье положение}
  \item \todo{Четвертое положение}
\end{enumerate}

%В папке Documents можно ознакомиться в решением совета из Томского ГУ
%в файле \verb+Def_positions.pdf+, где обоснованно даются рекомендации
%по формулировкам защищаемых положений. 

{\reliability} обусловлена корректным оцениванием адекватности разработанной в рамках предложенного адаптивного алгоритма синхронизации, проверкой
свойств и вычислительной сложности разработанных алгоритмов, а также подтверждена результатами экспериментов.


{\probation}
Основные результаты работы докладывались~на:
перечисление основных конференций, симпозиумов и~т.\:п.

{\contribution} Автор принимал активное участие \ldots

%\publications\ Основные результаты по теме диссертации изложены в ХХ печатных изданиях~\cite{Sokolov,Gaidaenko,Lermontov,Management},
%Х из которых изданы в журналах, рекомендованных ВАК~\cite{Sokolov,Gaidaenko}, 
%ХХ --- в тезисах докладов~\cite{Lermontov,Management}.

\ifnumequal{\value{bibliosel}}{0}{% Встроенная реализация с загрузкой файла через движок bibtex8
    \publications\ Основные результаты по теме диссертации изложены в XX печатных изданиях, 
    X из которых изданы в журналах, рекомендованных ВАК, 
    X "--- в тезисах докладов.%
}{% Реализация пакетом biblatex через движок biber
%Сделана отдельная секция, чтобы не отображались в списке цитированных материалов
    \begin{refsection}%
        \printbibliography[heading=countauthornotvak, env=countauthornotvak, keyword=biblioauthornotvak, section=1]%
        \printbibliography[heading=countauthorvak, env=countauthorvak, keyword=biblioauthorvak, section=1]%
        \printbibliography[heading=countauthorconf, env=countauthorconf, keyword=biblioauthorconf, section=1]%
        \printbibliography[heading=countauthor, env=countauthor, keyword=biblioauthor, section=1]%
        \publications\ Основные результаты по теме диссертации изложены в \arabic{citeauthor} печатных изданиях\nocite{bib1,bib2}, 
        \arabic{citeauthorvak} из которых изданы в журналах, рекомендованных ВАК\nocite{vakbib1,vakbib2}, 
        \arabic{citeauthorconf} "--- в тезисах докладов\nocite{confbib1,confbib2}.
    \end{refsection}
}
При использовании пакета \verb!biblatex! для автоматического подсчёта
количества публикаций автора по теме диссертации, необходимо
их здесь перечислить с использованием команды \verb!\nocite!.
    

